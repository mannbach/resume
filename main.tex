% ========================================================
% This document is a customizable CV/Resume template built using LaTeX.
% The template is designed for easy customization and clear structure.
%
% Author: Matthew DeVerna (www.matthewdeverna.com)
% Date: 2024
% Design: Cased on Hause Lin's CV (hauselin.com)
% 
% Project Overview:
% -----------------
% This LaTeX document is designed to help you create a professional CV or 
% resume with ease. It uses as little fancy LaTex functionality or custom functions as possible to maximize its longterm durability and flexibility.
% The document is structured into multiple sections, each loaded from 
% separate subfiles for modularity and ease of maintenance. 
%
% Key Features:
% -------------
% - Customizable sections: Education, Research Experience, Awards, Publications, etc.
% - Bookmarks in the PDF for easy navigation
% - Styled bibliography with BibLaTeX
% - Hyperlinked email and website
% - FontAwesome icons for additional styling
%
% Getting Started:
% ----------------
% 1. Customize your personal information by modifying the \mytitle command.
% 2. Add your content to the respective subfiles (e.g., education.tex, exp_research.tex).
% 3. Update the bibliography file (ref.bib) with your publications and categorize them with keywords.
%
% Important Notes:
% ----------------
% - This main file includes the overall structure and settings. 
% - Each section has its own detailed instructions for further customization.
%
% ========================================================

\documentclass[11pt]{article} % Choose the document class and font size
\usepackage[margin=1in]{geometry} % Page layout settings

% Set the citation style
\usepackage[
    backend=biber,      % Specifies the backend to be used by BibLaTeX for processing the bibliography. 'biber' is the default backend.
    maxnames=20,        % Limits the maximum number of author names to display before abbreviating with "et al."
    style=nature,       % Sets the citation style to 'nature,' which is commonly used in scientific papers.
    sorting=ydnt,       % Specifies the sorting order of entries in the bibliography:
                        % y - year (descending)
                        % d - descending order
                        % n - name
                        % t - title
    defernumbers=true,  % Delays the assignment of citation numbers until the end of the document, allowing for the correct order of citations within each bibliography section.
]{biblatex}
\addbibresource{ref.bib} % Adds the bibliography resource file 'ref.bib' containing all the references.

% Allows columns that stretch across pages
\usepackage{longtable}

% Table functionality and beautification (not strictly needed)
\usepackage{bookmark}

% Use icons, if you want.
% All available icons: http://mirrors.ibiblio.org/CTAN/fonts/fontawesome5/doc/fontawesome5.pdf
\usepackage{fontawesome}

% Allows font justification control (needed for clean pub-list formatting)
\usepackage{ragged2e}

% For underlining with line breaks
\usepackage{soul} 

% All fonts: https://tug.org/FontCatalogue/
\usepackage{kpfonts} % More professional font
% \usepackage[default]{sourcecodepro} % Code-like font
\usepackage[T1]{fontenc}

% Control hyperlinks and colors
% CUSTOM COLORS INCLUDED DIRECTLY AFTER \begin{document}
\usepackage{xcolor}
\usepackage{hyperref}
\hypersetup{
    colorlinks=true,        % Enable colored links
    breaklinks=true,        % Allow links to break across lines
    linkcolor=cornflowerblue,    % Color of internal links
    urlcolor=cornflowerblue,     % Color of URL links
    anchorcolor=cornflowerblue,  % Color of anchors
    citecolor=cornflowerblue,    % Color of citations
    pdftitle={Your Title},    % Title of the PDF
    pdfauthor={Your Name}, % Author of the PDF
    bookmarksopen=true,      % Open bookmarks panel at start
}

%%% CONVENIENCE FUNCTIONS GO HERE %%%
%%% ----------------------------- %%%
\newcommand{\mytitle}[4]{
  \begin{center}
    \Large\textbf{#1}\normalsize \\ % Name in large bold font
    \href{mailto:#2}{#2} \\ % Email with mailto: link
    \href{https://#3}{#3} \\ % Website with link
    #4 % Address
  \end{center}
}
%%% ----------------------------- %%%


\begin{document}
% Set custom colors here (imported directly after \begin{document})
% The below use HTML hex codes.
% More HTML hex codes: https://encycolorpedia.com/html
\definecolor{firebrick}{HTML}{b22222} 
\definecolor{darkslategrey}{HTML}{2f4f4f} 
\definecolor{cornflowerblue}{HTML}{6495ed} 
\definecolor{mediumslateblue}{HTML}{7b68ee}  % Load custom colors from colors file
\mytitle{Your Name}{person@gmail.com}{yourwebsite.com}{Your Address\\Goes here\\This field can be excluded} % Insert your custom title


% Ensure right side margin is not surpassed by bibliography and the right margin is aligned throughout
\RaggedRight


% These \pdfbookmark lines create bookmarks in the exported PDF document that display in the left pane.
% Value in [] sets the indentation level of the bookmark
\pdfbookmark[1]{Education}{}
\section*{Education}
% Add your educational background here!

% NOTE: If you want to remove the "Expected" footnote, you will want to remove:
% - Directly below: \renewcommand, \setcounter
% - In the table: \footnotemark in the left column
% - After the table: \footnotetext, \renewcommand, \setcounter

% Different numbers in "\setcounter{footnote}{0}" use different symbols

\begin{longtable}[l]{@{}p{.15\textwidth} p{0.85\textwidth}}
    Since 09.2021 & Network Science \textbf{Ph.D.} candidate\\
                  & \href{https://networkdatascience.ceu.edu/}{Department of Network and Data Science} of Central European University \\

    Since 04.2021 & \textbf{Ph.D.} candidate, \href{https://networkinequality.com/}{Network Inequality} at \href{https://csh.ac.at}{Complexity Science Hub}~\cite{bachmann.etal_cumulativeadvantagebrokerage_2024,neuhauser.etal_improvingvisibilityminorities_2023}\\

    2016--2021 & \textbf{M.Sc.} in \href{https://www.informatik.rwth-aachen.de/}{Computer Science}, RWTH Aachen University, Germany\vspace{-1em}
               \begin{itemize}
                   \item Measuring temporal Dependencies in Music Listening Behavior (Thesis)\vspace{-1em}
                   \item Behavioral Analysis of Political Users of Reddit (Seminar)
               \end{itemize}\vspace{-2em}\\
               & Erasmus+ Exchange Semester, Aalto University, Helsinki, Finland\\
               & Student Assistant, \href{https://cssh.rwth-aachen.de/}{CSSH}, RWTH Aachen~\cite{schumacher.etal_effectsrandomnessstability_2020}\\

    2013--2016 & \textbf{B.Sc.} in \href{https://www.fh-aachen.de/en/studies/degree-programmes/applied-mathematics-and-computer-science-dual-bsc}{Scientific Programming}, FH Aachen, Germany\\



% \end{tabularx}
\end{longtable}


\pdfbookmark[1]{Research Experience}{exp_research}
\section*{Research Experience}
\label{exp_research}
% List your research experience here

\begin{longtable}[l]{@{}p{.125\textwidth} p{0.875\textwidth}}

    2020-- & Research Assistant, \href{https://example.com/}{Cool Lab}, Your University, (Advisor: \href{https://example.com/}{Dr. Thinks-a-lot}) \\

    2020-- & Research Assistant, \href{https://example.com/}{Fancy Science Lab}, Your University, (Advisor: \href{https://example.com/}{Dr. Math Wiz}) \\

\end{longtable}

\pdfbookmark[1]{Awards \& Honors}{awards}
\section*{Awards \& Honors}
\label{awards}
% Include awards, honors, grant funding, etc. here

\begin{longtable}[l]{@{}p{.125\textwidth} p{0.875\textwidth}}
    % Use custom symbol footnote for "expected"
    07.2025 & \href{https://www.oefg.at/foerderungen/internationale-kommunikation/}{\emph{Internationale Kommunikation}} (\"Osterreichische Forschungsgemeinschaft travel grant)\\
    04.2024 & \href{https://www.accelnet-multinet.org/}{\emph{AccelNet-Multinet}-fellowship} (US research visit award)\\
    11.2023 & \href{https://stipendien.oeaw.ac.at/en/fellowships/doc}{\emph{DOC}-fellowship} (\textbf{2-year stipend}) awarded by the Austrian Academy of Sciences
    \begin{itemize}
        \item \raggedright Network Inequality and its Effect on the Glass Ceiling Phenomenon: How Gender Biases in Collaboration could explain Women’s Underrepresentation in Top Ranks
    \end{itemize}

\end{longtable}



\pdfbookmark[1]{Publications}{pubs}
\section*{Publications}
\label{pubs}

% Add equal contribution dagger
\vspace{-.75em}
\small
\faGoogle~\href{https://scholar.google.com/}{Google Scholar}\\
$\dagger \rightarrow$ Equal contribution
\normalsize


\pdfbookmark[2]{Journal Articles}{journal-article}
\subsection*{Journal Articles}
\label{journal-article}
\newrefcontext[labelprefix=J] % Will prefix bibliography numbers with this letter
% Ensures publications which are not cited in the document are included in the above sections
\nocite{*} % Ensures uncited items are included
\printbibliography[
    type=article, % Only include @article ref.bib items
    heading=none, % Do not include header. Gives us more control.
    resetnumbers=true, % Start item counter from zero
    keyword=J % Include items in ref.bib with keyword={J}
]

\pdfbookmark[2]{Peer-reviewed Conference Proceedings}{conferences}
\subsection*{Peer-reviewed Conference Proceedings}
\label{conferences}
\newrefcontext[labelprefix=C]
\printbibliography[type=inproceedings,heading=none,resetnumbers=true,keyword=C]

\pdfbookmark[2]{Working papers}{working-papers}
\subsection*{Working papers}
\label{working-papers}
\newrefcontext[labelprefix=W]
\printbibliography[type=misc,heading=none,resetnumbers=true,keyword=R]
    

\pdfbookmark[1]{Tools \& Software}{tools}
\section*{Tools \& Software}
\label{tools}
\subsection*{\href{https://example.com/}{\textbf{Group Name}}}
\begin{itemize}
    \item[] \href{https://example.com}{Cool dashboard}: A dashboard that displays data in a really awesome way.
    
    \item[] \href{https://example.com}{monkeytools}: Python package for the everyday codemonkey (\href{https://pypi.org}{PyPi} | \href{https://github.com/}{GitHub}). Demonstrated at (\href{https://pycon.org/}{PyCon 2024}).
    

\end{itemize}


\subsection*{Group 2}

\begin{itemize}
    \item[] Other stuff goes here.
\end{itemize}




\pdfbookmark[1]{Presentations}{presentations}
\section*{Presentations}
\label{presentations}

% Include any additional details here
% \vspace{-.75em}
% \small
% $\dagger \rightarrow$ Equal contribution
% \normalsize

\pdfbookmark[2]{Talks}{talks}
\subsection*{Talks}
\label{talks}
\newrefcontext[labelprefix=T]
\printbibliography[type=misc,heading=none,resetnumbers=true,keyword=T]

\pdfbookmark[2]{Posters}{posters}
\subsection*{Posters}
\label{posters}
\newrefcontext[labelprefix=P]        
\printbibliography[type=misc,heading=none,resetnumbers=true,keyword=P]

\pdfbookmark[2]{Demonstrations \& Tutorials}{demos}
\subsection*{Demonstrations \& Tutorials}
\label{demos}
\newrefcontext[labelprefix=D] \printbibliography[type=misc,heading=none,resetnumbers=true,keyword=D]


\pdfbookmark[1]{Selected Media Coverage}{media}
\section*{Selected Media Coverage}
\label{media}
%  Include media coverage here
% 
\begin{longtable}[l]{@{}p{.125\textwidth} p{0.875\textwidth}}

    2024 & The New York Times, \href{https://nyt.com}{Scientists figure out the meaning of life.}~\cite{art1} (Note how the regular citation functions work throughout the document!) \\
    
    2024 & AP News, \href{https://apnews.com/}{What even is science?}~\cite{art2} (Point to any project in your ref.bib file with the standard `cite' function) \\

    
\end{longtable}

 % Input lines load the material from the subdocuments


\pdfbookmark[1]{Teaching}{teaching}
\section*{Teaching}
\label{teaching}

\subsection*{University1 Name}
% 
\begin{longtable}[l]{@{}p{.125\textwidth} p{0.875\textwidth}}

    2024 & Assistant Instructor, Course Name (CODE 606) \\
    
    2024 & Guest lecturer, Science Theory and Practice (CODE 360) \\

\end{longtable}


\subsection*{University2 Name}
% 
\begin{longtable}[l]{@{}p{.125\textwidth} p{0.875\textwidth}}

    2024 & Assistant Instructor, Course Name (CODE 123) \\
    
    2024 & Guest lecturer, Science Theory and Practice (CODE 321) \\

\end{longtable}


\pdfbookmark[1]{Academic Advising}{advising}
\section*{Academic Advising}
\label{advising}
% Including academic advisement history here


% Remove \subsection{} lines and multiple tables if you only need one section!
\subsection*{Graduate}
\begin{longtable}[l]{@{}p{.125\textwidth} p{0.875\textwidth}}

    2023 & Name, University, Additional details \\

\end{longtable}

\subsection*{Undergraduate}
\begin{longtable}[l]{@{}p{.125\textwidth} p{0.875\textwidth}}

    2023 & Name, University, Additional details \\

\end{longtable}


\pdfbookmark[1]{Academic Service}{service}
\section*{Academic Service}
\label{service}


\subsection*{Guest Editor}
% 
\begin{longtable}[l]{@{}p{.125\textwidth} p{0.875\textwidth}}


    2023 & \href{https://example.com}{Link to collection} \\

% \end{tabularx}
\end{longtable}


\subsection*{Journal Reviewer}
% 
\begin{longtable}[l]{@{}p{.125\textwidth} p{0.875\textwidth}}

    2023 & Nature\\
    2023 & Science \\
    
% \end{tabularx}
\end{longtable}





\pdfbookmark[1]{Other Experience}{exp_other}
\section*{Other Experience}
\label{exp_other}
% Add other experience you don't know where to classify here

\begin{longtable}[l]{@{}p{.125\textwidth} p{0.875\textwidth}}
    2016--2019 & Student Assistant, Ericsson, Aachen, Germany\\
    & \begin{itemize}
        \item \raggedright Development of web-based dashboard to monitor software test results
    \end{itemize}\\
    2013--2016 & Apprenticeship Techno-Mathematical Software Developer (dual studies)\\
               & Ericsson, Aachen, Germany\\
    & \begin{itemize}
        \item \raggedright Focus on IT security and web development
    \end{itemize}
\end{longtable}


% Pretty ending with the date last updated
\centering
\rule{0.25\linewidth}{0.4pt}\\
\medskip
Last updated: \today

\end{document}
