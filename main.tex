% ========================================================
% This document is a customizable CV/Resume template built using LaTeX.
% The template is designed for easy customization and clear structure.
%
% Author: Matthew DeVerna (www.matthewdeverna.com)
% Date: 2024
% Design: Cased on Hause Lin's CV (hauselin.com)
%
% Project Overview:
% -----------------
% This LaTeX document is designed to help you create a professional CV or
% resume with ease. It uses as little fancy LaTex functionality or custom functions as possible to maximize its longterm durability and flexibility.
% The document is structured into multiple sections, each loaded from
% separate subfiles for modularity and ease of maintenance.
%
% Key Features:
% -------------
% - Customizable sections: Education, Research Experience, Awards, Publications, etc.
% - Bookmarks in the PDF for easy navigation
% - Styled bibliography with BibLaTeX
% - Hyperlinked email and website
% - FontAwesome icons for additional styling
%
% Getting Started:
% ----------------
% 1. Customize your personal information by modifying the \mytitle command.
% 2. Add your content to the respective subfiles (e.g., education.tex, exp_research.tex).
% 3. Update the bibliography file (ref.bib) with your publications and categorize them with keywords.
%
% Important Notes:
% ----------------
% - This main file includes the overall structure and settings.
% - Each section has its own detailed instructions for further customization.
%
% ========================================================

\documentclass[11pt]{article} % Choose the document class and font size
\usepackage[margin=1in]{geometry} % Page layout settings

% Set the citation style
\usepackage[
    backend=biber,      % Specifies the backend to be used by BibLaTeX for processing the bibliography. 'biber' is the default backend.
    maxnames=20,        % Limits the maximum number of author names to display before abbreviating with "et al."
    style=nature,       % Sets the citation style to 'nature,' which is commonly used in scientific papers.
    sorting=ydnt,       % Specifies the sorting order of entries in the bibliography:
                        % y - year (descending)
                        % d - descending order
                        % n - name
                        % t - title
    defernumbers=true,  % Delays the assignment of citation numbers until the end of the document, allowing for the correct order of citations within each bibliography section.
]{biblatex}
\addbibresource{ref.bib} % Adds the bibliography resource file 'ref.bib' containing all the references.

% Allows columns that stretch across pages
\usepackage{longtable}

% Table functionality and beautification (not strictly needed)
\usepackage{bookmark}

% Use icons, if you want.
% All available icons: http://mirrors.ibiblio.org/CTAN/fonts/fontawesome5/doc/fontawesome5.pdf
\usepackage{fontawesome5}

% Allows font justification control (needed for clean pub-list formatting)
\usepackage{ragged2e}

% For underlining with line breaks
\usepackage{soul}

% All fonts: https://tug.org/FontCatalogue/
\usepackage{kpfonts} % More professional font
% \usepackage[default]{sourcecodepro} % Code-like font
\usepackage[T1]{fontenc}

% Control hyperlinks and colors
% CUSTOM COLORS INCLUDED DIRECTLY AFTER \begin{document}
\usepackage{xcolor}
\usepackage{hyperref}
\hypersetup{
    colorlinks=true,        % Enable colored links
    breaklinks=true,        % Allow links to break across lines
    linkcolor=cornflowerblue,    % Color of internal links
    urlcolor=cornflowerblue,     % Color of URL links
    anchorcolor=cornflowerblue,  % Color of anchors
    citecolor=cornflowerblue,    % Color of citations
    pdftitle={Your Title},    % Title of the PDF
    pdfauthor={Your Name}, % Author of the PDF
    bookmarksopen=true,      % Open bookmarks panel at start
}

\begin{document}
\include{colors.tex} % Load custom colors from colors file

\begin{center}
  \Large\textbf{Jan Bachmann}\normalsize
\end{center}

\begin{tabular}{ccccc}
  \faAt[regular]~\href{mailto:jan@mannbach.de}{jan@mannbach.de} &
  \faHome[regular]~\href{https://mannbach.de}{mannbach.de} &
  \faOrcid~\href{https://orcid.org/0000-0002-6153-4714}{ORCiD} &
  \faGoogle~\href{https://scholar.google.de/citations?user=NkxVbcUAAAAJ&hl=de}{Google Scholar} &
  \faGithub~\href{https://github.com/mannbach}{GitHub} \\
\end{tabular}


% Ensure right side margin is not surpassed by bibliography and the right margin is aligned throughout
\RaggedRight


% These \pdfbookmark lines create bookmarks in the exported PDF document that display in the left pane.
% Value in [] sets the indentation level of the bookmark
\pdfbookmark[1]{Research Experience}{}
\section*{Education}
% Add your educational background here!

% NOTE: If you want to remove the "Expected" footnote, you will want to remove:
% - Directly below: \renewcommand, \setcounter
% - In the table: \footnotemark in the left column
% - After the table: \footnotetext, \renewcommand, \setcounter

% Different numbers in "\setcounter{footnote}{0}" use different symbols

\begin{longtable}[l]{@{}p{.15\textwidth} p{0.85\textwidth}}
    01.--05.2025 & \textbf{Research visit} at University of Colorado, Boulder\\
                & with Aaron Clauset \href{https://aaronclauset.github.io/}{\faGlobe} and Daniel Larremore \href{https://larremorelab.github.io/people/}{\faGlobe}\\
    Since 09.2021 & Network Science \textbf{Ph.D.} candidate\\
                  & Department of Network and Data Science of Central European University \\

    Since 04.2021 & \textbf{Ph.D.} candidate, \href{https://networkinequality.com/}{Network Inequality} at Complexity Science Hub~\cite{bachmann.etal_patch_2025,bachmann.etal_cumulativeadvantagebrokerage_2024,zappala.etal_genderdisparitiesdissemination_2024,she.etal_genderdifferencescollaboration_2024,neuhauser.etal_improvingvisibilityminorities_2023}
        \begin{itemize}
            \item \raggedright Understanding the Emergence of Inequalities from Link Formations and Rank Dynamics in Collaboration Networks (Thesis supervised by Fariba Karimi \href{https://csh.ac.at/fariba-karimi/}{\faGlobe})
        \end{itemize}\\

    2019--2020 & Student Assistant, CSSH, RWTH Aachen~\cite{schumacher.etal_effectsrandomnessstability_2020}\\

    09.--12.2018       & Erasmus+ Exchange Semester, Aalto University, Helsinki, Finland\\

    2016--2021 & \textbf{M.Sc.} in Computer Science, RWTH Aachen University, Germany
               \begin{itemize}
                   \item \raggedright Measuring Temporal Dependencies in Music Listening Behavior (Thesis supervised by Markus Strohmaier~\href{https://www.bwl.uni-mannheim.de/en/information-systems/chairs/prof-dr-strohmaier/}{\faGlobe})
                   \item \raggedright Behavioral Analysis of Political Reddit Users (Seminar contribution to~\href{https://dl.acm.org/doi/abs/10.1145/3342220.3343662}{later publication})
               \end{itemize}\\

    2013--2016 & \textbf{B.Sc.} in \href{https://www.fh-aachen.de/en/studies/degree-programmes/applied-mathematics-and-computer-science-dual-bsc}{Scientific Programming}, FH Aachen, Germany\\



% \end{tabularx}
\end{longtable}


\pdfbookmark[1]{Awards \& Honors}{awards}
\section*{Awards}
\label{awards}
% Include awards, honors, grant funding, etc. here

\begin{longtable}[l]{@{}p{.125\textwidth} p{0.875\textwidth}}
    % Use custom symbol footnote for "expected"
    11.2023 & \href{https://stipendien.oeaw.ac.at/en/fellowships/doc}{\emph{DOC}-fellowship} (2-year stipend) awarded by the Austrian Academy of Sciences\vspace{-0.5em}
    \begin{itemize}
        \item Network Inequality and its Effect on the Glass Ceiling Phenomenon: How Gender Biases in Collaboration could explain Women’s Under-Representation in Top Ranks
    \end{itemize}

\end{longtable}



\pdfbookmark[1]{Publications}{pubs}
\section*{Publications}
\label{pubs}

% Add equal contribution dagger
\vspace{-.75em}
\small

\hspace{1em}

\normalsize
\pdfbookmark[2]{Journal Articles}{journal-article}
\nocite{*} % Ensures uncited items are included
% Ensures publications which are not cited in the document are included in the above sections
\printbibliography[
    heading=none, % Do not include header. Gives us more control.
    resetnumbers=true, % Start item counter from zero
]

\pdfbookmark[1]{Presentations}{presentations}
\section*{Presentations}
\label{presentations}

\pdfbookmark[2]{Talks}{talks}
\subsection*{Talks}

\pdfbookmark[2]{Hosting}{hosting}
\subsection*{Hosting}
\label{hosting}
% \newrefcontext[labelprefix=D] \printbibliography[type=misc,heading=none,resetnumbers=true,keyword=D]
\begin{longtable}[l]{@{}p{.15\textwidth} p{0.85\textwidth}}
  09.2024     & \href{https://sites.google.com/view/snma-tutorial/2024}{“Synthetic Network Data”} tutorial at ECML PKDD\\
  07.2023     & \href{https://sites.google.com/view/netin-satellite-2023/home}{"Network Inequality"} satellite at NetSci 2023\\
\end{longtable}

\pdfbookmark[1]{Academic Services}{services}
\section*{Academic Service}
\label{service}
\subsection*{Teaching}
\begin{longtable}[l]{@{}p{.125\textwidth} p{0.875\textwidth}}
01.–03.2023 & Teaching assistant for \emph{Data Management}, Central European University\\
\end{longtable}


\subsection*{Workshops}
\begin{longtable}[l]{@{}p{.125\textwidth} p{0.875\textwidth}}
    07.2022     & Summer school on \emph{Data Science: Models, Algorithms, AI and Beyond}, Lipari, Italy\\
    02.2022     & ACM-W Winter School on Fairness in AI, Greece (online)\\
    01.2022	    & Winter Workshop on Complex Systems, Besançon, France\\
\end{longtable}

\subsection*{Reviewing}
%
\begin{longtable}[l]{@{}p{.15\textwidth} p{0.85\textwidth}}
    2023--2024 & International Conference on Computational Social Science (IC$^2$S$^2$) \\
    2022 & Nature Communications\\
    2022 & Data Mining and Knowledge Discovery\\
\end{longtable}




\pdfbookmark[1]{Professional Experience}{exp_prof}
\section*{Other Experience}
\label{exp_other}
% Add other experience you don't know where to classify here

\begin{longtable}[l]{@{}p{.125\textwidth} p{0.875\textwidth}}
    2016--2019 & Student Assistant, Ericsson, Aachen, Germany\\
    2013--2016 & Apprenticeship Mathematical- technical Software Developer (dual studies)\\
               & Ericsson, Aachen, Germany


\end{longtable}


% Pretty ending with the date last updated
\centering
\rule{0.25\linewidth}{0.4pt}\\
\medskip
Last updated: \today

\end{document}
