% ========================================================
% This document is a customizable CV/Resume template built using LaTeX.
% The template is designed for easy customization and clear structure.
%
% Author: Matthew DeVerna (www.matthewdeverna.com)
% Date: 2024
% Design: Cased on Hause Lin's CV (hauselin.com)
%
% Project Overview:
% -----------------
% This LaTeX document is designed to help you create a professional CV or
% resume with ease. It uses as little fancy LaTex functionality or custom functions as possible to maximize its longterm durability and flexibility.
% The document is structured into multiple sections, each loaded from
% separate subfiles for modularity and ease of maintenance.
%
% Key Features:
% -------------
% - Customizable sections: Education, Research Experience, Awards, Publications, etc.
% - Bookmarks in the PDF for easy navigation
% - Styled bibliography with BibLaTeX
% - Hyperlinked email and website
% - FontAwesome icons for additional styling
%
% Getting Started:
% ----------------
% 1. Customize your personal information by modifying the \mytitle command.
% 2. Add your content to the respective subfiles (e.g., education.tex, exp_research.tex).
% 3. Update the bibliography file (ref.bib) with your publications and categorize them with keywords.
%
% Important Notes:
% ----------------
% - This main file includes the overall structure and settings.
% - Each section has its own detailed instructions for further customization.
%
% ========================================================

\documentclass[11pt]{article} % Choose the document class and font size
\usepackage[margin=1in]{geometry} % Page layout settings

% Set the citation style
\usepackage[
    backend=biber,      % Specifies the backend to be used by BibLaTeX for processing the bibliography. 'biber' is the default backend.
    maxnames=20,        % Limits the maximum number of author names to display before abbreviating with "et al."
    style=nature,       % Sets the citation style to 'nature,' which is commonly used in scientific papers.
    sorting=ydnt,       % Specifies the sorting order of entries in the bibliography:
                        % y - year (descending)
                        % d - descending order
                        % n - name
                        % t - title
    defernumbers=true,  % Delays the assignment of citation numbers until the end of the document, allowing for the correct order of citations within each bibliography section.
]{biblatex}
\addbibresource{Publications.bib} % Adds the bibliography resource file 'ref.bib' containing all the references.

% Allows columns that stretch across pages
\usepackage{longtable}

% Table functionality and beautification (not strictly needed)
\usepackage{bookmark}

% Use icons, if you want.
% All available icons: http://mirrors.ibiblio.org/CTAN/fonts/fontawesome5/doc/fontawesome5.pdf
\usepackage{fontawesome5}

% Allows font justification control (needed for clean pub-list formatting)
\usepackage{ragged2e}

% For underlining with line breaks
\usepackage{soul}

% All fonts: https://tug.org/FontCatalogue/
\usepackage{kpfonts} % More professional font
% \usepackage[default]{sourcecodepro} % Code-like font
\usepackage[T1]{fontenc}

% https://ctan.fisiquimicamente.com/fonts/academicons/academicons.pdf
% \usepackage{academicons}

% Control hyperlinks and colors
% CUSTOM COLORS INCLUDED DIRECTLY AFTER \begin{document}
\usepackage{xcolor}
\usepackage{hyperref}
\hypersetup{
    colorlinks=true,        % Enable colored links
    breaklinks=true,        % Allow links to break across lines
    linkcolor=mediumslateblue,    % Color of internal links
    urlcolor=mediumslateblue,     % Color of URL links
    anchorcolor=mediumslateblue,  % Color of anchors
    citecolor=mediumslateblue,    % Color of citations
    pdftitle={Resum\'e Jan Bachmann},    % Title of the PDF
    pdfauthor={Jan Bachmann}, % Author of the PDF
    bookmarksopen=true,      % Open bookmarks panel at start
}

\begin{document}
% Set custom colors here (imported directly after \begin{document})
% The below use HTML hex codes.
% More HTML hex codes: https://encycolorpedia.com/html
\definecolor{firebrick}{HTML}{b22222} 
\definecolor{darkslategrey}{HTML}{2f4f4f} 
\definecolor{cornflowerblue}{HTML}{6495ed} 
\definecolor{mediumslateblue}{HTML}{7b68ee}  % Load custom colors from colors file

\begin{center}
  \Large\textbf{Jan Bachmann}\normalsize

  \vspace{1em}

  \begin{tabular}{ccccc}
    \faAt~\href{mailto:jan@mannbach.de}{jan@mannbach.de} &
    \faGlobe~\href{https://mannbach.de}{mannbach.de} &
    \faGoogle~\href{https://scholar.google.de/citations?user=NkxVbcUAAAAJ&hl=de}{Jan Bachmann} &
    \faGithub~\href{https://github.com/mannbach}{mannbach} &
    % \faOrcid~\href{https://orcid.org/0000-0002-6153-4714}{0000-0002-6153-4714}
    % Skills: \faPython~\faDocker
  \end{tabular}
\end{center}
%
% Ensure right side margin is not surpassed by bibliography and the right margin is aligned throughout
\RaggedRight
%
% These \pdfbookmark lines create bookmarks in the exported PDF document that display in the left pane.
% Value in [] sets the indentation level of the bookmark
\pdfbookmark[1]{Research Experience}{}
\section*{Education}
% Add your educational background here!

% NOTE: If you want to remove the "Expected" footnote, you will want to remove:
% - Directly below: \renewcommand, \setcounter
% - In the table: \footnotemark in the left column
% - After the table: \footnotetext, \renewcommand, \setcounter

% Different numbers in "\setcounter{footnote}{0}" use different symbols

\begin{longtable}[l]{@{}p{.15\textwidth} p{0.85\textwidth}}
    Since 09.2021 & Network Science \textbf{Ph.D.} candidate\\
                  & \href{https://networkdatascience.ceu.edu/}{Department of Network and Data Science} of Central European University \\

    Since 04.2021 & \textbf{Ph.D.} candidate, \href{https://networkinequality.com/}{Network Inequality} at \href{https://csh.ac.at}{Complexity Science Hub}~\cite{bachmann.etal_cumulativeadvantagebrokerage_2024,neuhauser.etal_improvingvisibilityminorities_2023}\\

    2016--2021 & \textbf{M.Sc.} in \href{https://www.informatik.rwth-aachen.de/}{Computer Science}, RWTH Aachen University, Germany\vspace{-1em}
               \begin{itemize}
                   \item Measuring temporal Dependencies in Music Listening Behavior (Thesis)\vspace{-1em}
                   \item Behavioral Analysis of Political Users of Reddit (Seminar)
               \end{itemize}\vspace{-2em}\\
               & Erasmus+ Exchange Semester, Aalto University, Helsinki, Finland\\
               & Student Assistant, \href{https://cssh.rwth-aachen.de/}{CSSH}, RWTH Aachen~\cite{schumacher.etal_effectsrandomnessstability_2020}\\

    2013--2016 & \textbf{B.Sc.} in \href{https://www.fh-aachen.de/en/studies/degree-programmes/applied-mathematics-and-computer-science-dual-bsc}{Scientific Programming}, FH Aachen, Germany\\



% \end{tabularx}
\end{longtable}


\pdfbookmark[1]{Academia}{academia}
\section*{Academia}
\label{academia}
\pdfbookmark[1]{Awards}{awards}
\subsection*{Awards}
\label{awards}
% Include awards, honors, grant funding, etc. here

\begin{longtable}[l]{@{}p{.125\textwidth} p{0.875\textwidth}}
    % Use custom symbol footnote for "expected"
    07.2025 & \href{https://www.oefg.at/foerderungen/internationale-kommunikation/}{\emph{Internationale Kommunikation}} (\"Osterreichische Forschungsgemeinschaft travel grant)\\
    04.2024 & \href{https://www.accelnet-multinet.org/}{\emph{AccelNet-Multinet}-fellowship} (US research visit award)\\
    11.2023 & \href{https://stipendien.oeaw.ac.at/en/fellowships/doc}{\emph{DOC}-fellowship} (\textbf{2-year stipend}) awarded by the Austrian Academy of Sciences
    \begin{itemize}
        \item \raggedright Network Inequality and its Effect on the Glass Ceiling Phenomenon: How Gender Biases in Collaboration could explain Women’s Underrepresentation in Top Ranks
    \end{itemize}

\end{longtable}



\pdfbookmark[1]{Publications}{pubs}
\subsection*{Publications}
\label{pubs}

% Add equal contribution dagger
\small

\hspace{1em}

\normalsize
\pdfbookmark[2]{Journal Articles}{journal-article}
\nocite{*} % Ensures uncited items are included
% Ensures publications which are not cited in the document are included in the above sections
\printbibliography[
    heading=none, % Do not include header. Gives us more control.
    resetnumbers=true, % Start item counter from zero
]

\pdfbookmark[2]{Tools}{tools}
\subsection*{Tools}
%
\begin{itemize}
    \item \textbf{NetIn} (~\href{https://pypi.org/project/netin/}{\faPython}
        \hspace{0.5em}\href{https://github.com/CSHVienna/NetworkInequalities}{\faGithub}
        \hspace{0.5em}\href{https://cshvienna.github.io/NetworkInequalities/}{\faFile*[regular]}~)
        : A Python package for the analysis of network inequalities.

\end{itemize}

\pdfbookmark[2]{Talks}{talks}
\subsection*{Talks \& Posters}
\begin{longtable}[l]{@{}p{.15\textwidth} p{0.85\textwidth}}
  07.2025     & Poster at International Conference on Computational Social Science (IC$^2$S$^2$)\\
              & in Norrk\"oping, Sweden on~\cite{bachmann.etal_patch_2025}\\
  08.2024     & Visualizing Complexity Science Workshop; Vienna, Austria\\
  03.2024     & Talk at DPG Spring Convention, Berlin, Germany on~\cite{bachmann.etal_cumulativeadvantagebrokerage_2024}\\
  09.2023     & Talk at Symposium on Scientific Elites, Copenhagen, Denmark on~\cite{bachmann.etal_cumulativeadvantagebrokerage_2024}\\
  07.2023     & Talk at International Conference on Computational Social Science (IC$^2$S$^2$)\\
              & in Copenhagen, Denmark on~\cite{bachmann.etal_cumulativeadvantagebrokerage_2024}\\
  10.2022     & Talk at Conference on Complex Systems (CCS), Palma de Mallorca, Spain on~\cite{bachmann.etal_cumulativeadvantagebrokerage_2024}
\end{longtable}

\pdfbookmark[2]{Hosting}{hosting}
\subsection*{Organizing}
\label{organizing}
% \newrefcontext[labelprefix=D] \printbibliography[type=misc,heading=none,resetnumbers=true,keyword=D]
\begin{longtable}[l]{@{}p{.15\textwidth} p{0.85\textwidth}}
  09.2024     & \href{https://sites.google.com/view/snma-tutorial/2024}{\emph{Synthetic Network Data}} tutorial at ECML PKDD\\
  07.2023     & \href{https://sites.google.com/view/netin-satellite-2023/home}{\emph{Network Inequality}} satellite at NetSci 2023\\
\end{longtable}

\pdfbookmark[1]{Academic Services}{services}


\subsection*{Guest Editor}
% 
\begin{longtable}[l]{@{}p{.125\textwidth} p{0.875\textwidth}}


    2023 & \href{https://example.com}{Link to collection} \\

% \end{tabularx}
\end{longtable}


\subsection*{Journal Reviewer}
% 
\begin{longtable}[l]{@{}p{.125\textwidth} p{0.875\textwidth}}

    2023 & Nature\\
    2023 & Science \\
    
% \end{tabularx}
\end{longtable}




\pdfbookmark[1]{Professional Experience}{exp_prof}
\section*{Industry Experience}
\label{exp_other}
% Add other experience you don't know where to classify here

\begin{longtable}[l]{@{}p{.125\textwidth} p{0.875\textwidth}}
    2016--2019 & Student Assistant, Ericsson, Aachen, Germany\\
    & \begin{itemize}
        \item \raggedright Development of web-based dashboard to monitor software test results
    \end{itemize}\\
    2013--2016 & Apprenticeship Techno-Mathematical Software Developer (dual studies)\\
               & Ericsson, Aachen, Germany\\
    & \begin{itemize}
        \item \raggedright Focus on IT security and web development
    \end{itemize}
\end{longtable}


% Pretty ending with the date last updated
\centering
\rule{0.25\linewidth}{0.4pt}\\
\medskip
Last updated: \today

\end{document}
